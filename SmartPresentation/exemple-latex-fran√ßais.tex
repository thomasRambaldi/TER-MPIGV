\documentclass[english]{beamer}

\usepackage[utf8]{inputenc}
\usepackage[]{babel}

% Pour la première page
\title{Presentation Française}
\author{Kevin Lebreton}
\date{29/05/16}
		
\begin{document}
  \begin{frame}
 \frametitle{Exemple Français}
	% Nous allons commencer notre presentation, passons à l'introduction.
	    \titlepage
  \end{frame}

  
  \begin{frame}{Table des matières}
	%Nous sommes dans la seconde diapositive qui représente la table des matières    
	\tableofcontents
  \end{frame}

  \section{Notre projet}
  \begin{frame}{Titre de la troisieme diapositive}
    %Notre projet est un moteur de présentation intéractive guidée par la voix
    Moteur de présentation \\ 
    Controllé par la voix \\ 
    Projet et management.
  \end{frame}

  \section{Conclusion et exemples}
  \begin{frame}{Conclusion}
	%Le texte pour le premier bloc avec une animation
	\begin{block}{Nous pouvons utiliser des mots-clefs}<1->
	  	\begin{itemize}
	  		\item {\tt passer à la diapositive suivante}   		
    		\item {\tt passer à la diapositive précédante}
		\end{itemize}
	\end{block}

	%Le texte pour le second bloc avec une animation
	\begin{block}{Nous pouvons utiliser des mots-clefs spéciaux}<2->
	  	\begin{itemize}
	  		\item {\tt aller à la diapositive introduction}   		
    		\item {\tt aller à la diapositive conclusion}
		\end{itemize}
	\end{block}

	%Le texte pour le troisième bloc avec une animation
	\begin{block}{Nous pouvons dire notre texte}<3->
	  	\begin{itemize}
	  		\item {\tt Vous avez juste à lire le texte ci-dessous}   		
    		\item {\tt Le texte pour le troisième bloc avec une animation}
		\end{itemize}
	\end{block}
  \end{frame}
  
  {\1
\begin{frame}[plain,noframenumbering]
  \finalpage{Merci}
\end{frame}
}

\end{document}
